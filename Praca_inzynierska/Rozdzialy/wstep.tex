\section{Wstęp}
% Co w pracy znajdziemy
    % Skrót informacji potrzebnych do zrozumienia pracy
    % Działająca aplikacja multiplatformowa
    % Instrukcja utworzenia funkcjonującego piepline'a
% Czego w pracy nie znajdziemy
    % Dogłębnej analizy działania biblioteki do pathfindingu (wyłącznie linki)

Założeniem pracy jest stworzenie kompletnego rozwiązania w postaci \\ 
aplikacji ułatwiającej
nawigowanie po budynku Śląskiego Międzyuczelnianego Centrum Edukacji i Badań Interdyscyplinarnych (SMCEBI),
oraz zautomatyzowanego środowiska kompilującego i dystrybuującego rozwiązanie.

    W mojej pracy zawarłem krótkie objaśnienia terminów, których zrozumienie jest bardzo przydatne
(a czasem nawet kluczowe) do uruchomienia mojego rozwiązania. Najciekawsze źródła, których nie cytowałem,
ale odwiedziłem i naprowadziły mnie na poprawne tory bądź rozwijają temat umieściłem w rozdziale


    Podczas tworzenia programu skorzystałem z gotowej biblioteki do wyznaczania ścieżek,
wobec czego w niniejszej pracy znajduje się wyłącznie skrócone wytłumaczenie schematu działania
jej algorytmu.