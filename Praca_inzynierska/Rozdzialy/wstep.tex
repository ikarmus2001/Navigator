\section{Wstęp}

Założeniem pracy inżynierskiej jest objaśnienie zagadnienia oraz zaprezentowanie 
implementacji zautomatyzowanego procesu wydawczego, który będzie 
przygotowany na podstawie przykładowej aplikacji. 
Aby podkreślić jego u\-ni\-wer\-sal\-ność zdecydowałem się na skorzystanie z niżej opisanego frameworka 
.NET MAUI (rozdział \ref{mauiFrameworkSubsection}), który pozwala na wydanie ekwiwalentnej aplikacji na różne platformy.

W nowoczesnym biznesie i nauce istotne jest przyspieszanie, automatyzowanie 
oraz idąca za tym redukcja kosztów tworzenia i utrzymywania oprogramowania.
W~dobie wszechobecnych komputerów oraz naszej zależności od wykonywanej przez 
nie pracy, dostarczane usługi i dostęp do informacji to wartościowe dobra, 
na których brak nie możemy sobie pozwolić.

Motywacją do przygotowania tej pracy były moje doświadczenia oraz przekonania - 
dbałość o jakość, stabilność i przewidywalność wyników to cechy środowiska pracy, 
które powinny zajmować najwyższe miejsca na liście zawodowych priorytetów.
Istotność automatyzacji w obszarach ludzkiej niedoskonałości wiedzie do 
ogólnej poprawy efektów pracy, dzięki czemu będziemy działać wydajniej, 
a ilość czasu poświęconego na niepotrzebne napięcia i zdenerwowanie będziemy mogli 
zredukować na rzecz rozwoju i przyjemności.

Wierzę, że zapoznanie się z technikami i narzędziami zaprezentowanymi w tej pracy 
pozwoli zrozumieć oraz wdrożyć je do swojego środowiska programistycznego.

Najciekawsze źródła, których nie cytowałem, ale naprowadziły mnie na poprawne tory lub rozwijają temat, 
umieściłem w rozdziale~\ref{dodatkoweZrodla}.

