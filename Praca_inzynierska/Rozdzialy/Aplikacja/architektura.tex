\subsection{Architektura aplikacji}

\subsubsection{Wzorce projektowe}

\paragraph{Wzorzec Budowniczy -}
Tworzenie obiektu mapy postanowiłem rozwiązać za pomocą wzorca Buildera - 
specjalna klasa MapBuilder udostępnia metody, które pozwalają inkrementacyjnie 
dodawać coraz to nowe szczegóły do mapy, a następnie zwrócić gotowy widok, 
który został już przetłumaczony do HTML~\cprotect\footnote{%
    W aktualnym kształcie aplikacji narzucony jest zwracany typ \verb|string|, 
    ale w przyszłości może zostać zaimplementowana obsługa natywnych widoków,
    przez co typ zostanie odpowiednio zmieniony.}.

\paragraph{Wzorzec MVVM -}
W aplikacji starałem się korzystać ze wzorca projektowego MVVM (rozdział \ref{mvvmSubsection}) - 
wprowadziłem podział na odpowiednio:
\begin{itemize}
    \item Pliki znaczników - strona\verb|.xaml| wraz z code-behind 
    (kodu związanego bezpośrednio z danym widokiem) - strona\verb|.xaml.cs|
    \item ViewModel, przygotowany dla każdego widoku - strona\verb|ViewModel.cs|
    \item Modele, przechowywane w specjalnym folderze \verb|Models| oraz ujęte w odrębnej przestrzeni nazw
\end{itemize}

Zamysł polega na uniemożliwieniu bezpośrednich interakcji danych 
modelowych z widokami, a poprzez ViewModele. Takie podejście umożliwia bezpieczną zmianę 
widoków, bez zastanawiania się czy wpłyną one na zachowanie ViewModelu lub co gorsza wprost na dane.
Na Rys.~\ref{img:shortClassDiagram} widzimy skrótowy podział rozwiązania na klasy, pogrupowany 
zgodnie z ich przeznaczeniem oraz najważniejszą relacją z klasami sąsiadującymi.
\clearpage

\newgeometry{top=2cm,right=1cm}

\begin{figure}[!hb]
    \centering
    \includegraphics[width=0.9\textwidth]{shortClassDiagram.png}
    \caption{Diagram najważniejszych klas w projekcie SMCEBI Navigator}
    \label{img:shortClassDiagram}
\end{figure}

\subsubsection{Podział na projekty}
\begin{figure}[!ht]
    \centering
    \frame{\includegraphics[width=0.55\textwidth]{projectDiagram.png}}
    \caption{Podział rozwiązania na projekty}
    \label{img:projectDiagram}
\end{figure}

\newpage
\restoregeometry

Zgodnie z Rys.~\ref{img:projectDiagram}, oprócz bezpośredniego kodu aplikacji, 
zdecydowałem się na wyodrębnienie 3 projektów:

\paragraph{NavigatorTests -}
Projekt służący do statycznej analizy kodu poprzez prze\-pro\-wa\-dze\-nie testów jednostkowych.
Nie jest on obszerny, ale wraz z rozwojem oprogramowania może znaleźć się w nim więcej 
scenariuszy. Na potrzeby tej pracy został zaimplementowany jeden test pokazowy, 
żeby móc skonfigurować ich wywołanie za pomocą pipeline'a.

\paragraph{MapBuilder\_API\_Base -}
Projekt definiuje interfejs \verb|IMapBuilder|, który należy zaimplementować przy dodawaniu 
do aplikacji biblioteki wyświetlającej mapę, tak jak stało się to w przypadku LeafletAPI.
Ten podział wprowadza kompatybilność i możliwość zmian minimalnym kosztem programistycznym.

W obecnej wersji dodanie biblioteki wymaga dodania projektu do rozwiązania, specjalnej opcji w enumeratorze 
\verb|MapEngine| oraz dodanie odpowiedniej metody do \verb|ConfigParser|. Wraz z rozwojem aplikacji 
planuję ten proces skrócić do dołączenia biblioteki poprzez mechanizm refleksji~\cprotect\footnote{%
    Specjalna biblioteka \verb|System.Reflection| pozwala na dynamiczne włączanie do programu 
    dodatkowych bibliotek \verb|.dll| w trakcie uruchomienia aplikacji, co pozwala na rozbudowanie 
    funkcji programu. Taki system może być skomplikowany oraz niebezpieczny, więc zdecydowałem się 
    na odsunięcie tej decyzji w czasie.
}.

\paragraph{LeafletAPI -}
W projekcie zaimplementowałem generowanie strony HTML z wykorzystaniem wspomnianej biblioteki Leaflet~\cite{leafletGithub}.
Biblioteka tworzy obiekt \verb|MapBuilder| implementujący interfejs \verb|IMapBuilder|, który jest 
za pomocą aplikacji poszerzany o kolejne piętra, pokoje oraz obiekty z kreatora, a po wywołaniu 
metody \verb|Build| zwraca wynikowy skrypt, który zostaje zwrócony do aplikacji.

W zrozumieniu działania biblioteki, nieopisaną pomocą okazał się projekt Floorplans~\cite{floorplansGithub},
na którego podstawie byłem w stanie wstępnie sprawdzić zachowanie przeglądarkowych kontrolek, 
a następnie za pomocą inżynierii wstecznej zaimplementować jego funkcjonalność w uniwersalny sposób.


