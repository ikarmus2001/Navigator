\subsection{MVVM}
MVVM to akronim Model-View-Viewmodel (w tłumaczeniu Model - Widok - Model widoku), czyli programistycznego wzorca projektowego. \\
Polega na wprowadzeniu separacji pomiędzy modelem danych i sposobem ich obróbki, widokami je reprezentującymi oraz operacjami pomiędzy nimi. \\
Jego główną zaletą jest możliwość oddzielenia ogólnych założeń programu od szczegółów integracji z platformą 
- w projekcie~\cite{Halaczkiewicz_Navigator_GitHub} jest to wyraźnie zarysowane dzięki przyjętej 
konwencji nazewniczej plików (postfix -Model, -ViewModel lub -View).
Na rysunku~\ref{fig:MVVM_diagram} możemy zaobserwować zależność komunikacji klas stworzonych z wykorzystaniem właśnie tego wzorca.

\begin{figure}[!htp]
    \centering
    \begin{tikzpicture}[
        minimum height=1cm,
        block/.style={rectangle, draw=black!60, fill=green!30, thick, rounded corners=2mm}
        ]
            \node[block] (Model) {Model};
            \node[block] (ViewModel) [right=of Model] {ViewModel};
            \node[block] (View) [right=of ViewModel] {View};
            
            \draw[->] (Model.east) -- node[anchor=east]{} (ViewModel.west);
            \draw[->] (ViewModel.west) -- node[anchor=east]{} (View.east);
    \end{tikzpicture}
    \caption{Zależności między klasami w MVVM~\cite{mvvm}}
    \label{fig:MVVM_diagram}
\end{figure}