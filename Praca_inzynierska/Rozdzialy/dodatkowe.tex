\section{Dodatkowe źródła i przydatne materiały} \label{dodatkoweZrodla}
W tej sekcji umieszczam przydatne linki wraz z krótkimi opisami ich zawartości,
które przyczyniły się w istotny sposób do poprawy jakości pracy, ale nie wypisałem ich w cytowaniach,
ze względu na brak bezpośredniego cytowania.

\begin{itemize}
    \item \href{https://www.youtube.com/playlist?list=PLdo4fOcmZ0oUBAdL2NwBpDs32zwGqb9DY}%
        {YouTube - .NET MAUI for Beginners} \\
        oficjalny poradnik, który wstępnie oprowadził mnie po zastosowaniach i obsłudze frameworka.
    \item \href{https://www.youtube.com/watch?v=s1grtSSIRVA}%
        {YouTube - Signing \& Versioning Android Apps} \\
        Przydatny poradnik do podpisywania aplikacji wewnątrz pipeline'a.
    \item \href{https://developer.android.com/tools/zipalign}%
        {ZipAlign - narzędzie Android SDK}\\
        Opis i sposób użycia narzędzia \verb|zipalign|
    \item \href{https://learn.microsoft.com/en-us/dotnet/framework/reflection-and-codedom/how-to-load-assemblies-into-the-reflection-only-context}%
        {Importowanie bibliotek za pomocą mechanizmu refleksji}\\
        Szybka prezentacja sposobu importowania dynamicznych bibliotek wewnątrz uruchomionej aplikacji
    \item \href{https://fonts.google.com/icons}%
        {Material Symbols - biblioteka darmowych ikon od Google}\\
        W mojej aplikacji skorzystałem z kilku ikon udostępnianych przez firmę posiadającą i stale 
        rozwijającą Androida, czyli właśnie Google. Mnogość możliwości ich personalizacji 
        oraz wygoda korzystania (możliwość pobrania całej paczki lub tylko jej części, 
        dostępność formatów wektorowych lub ich rastrowych odpowiedników) sprawiła, 
        że z przyjemnością użyję jej ponownie.
\end{itemize}
