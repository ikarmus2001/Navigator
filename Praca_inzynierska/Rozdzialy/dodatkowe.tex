\section{Dodatkowe źródła i przydatne materiały} \label{dodatkoweZrodla}
W tej sekcji umieszczam przydatne linki wraz z krótkimi opisami ich zawartości,
które przyczyniły się w istotny sposób do poprawy jakości pracy, ale nie wypisałem ich w cytowaniach,
ze względu na brak bezpośredniego cytowania.

\begin{itemize}
    \item YouTube - .NET MAUI for Beginners \\
        oficjalny poradnik, który wstępnie oprowadził mnie po zastosowaniach i ob\-słu\-dze frameworka. \\
        \href{youtube.com/playlist?list=PLdo4fOcmZ0oUBAdL2NwBpDs32zwGqb9DY}%
        {youtube.com/playlist?list=PLdo4fOcmZ0oUBAdL2NwBpDs32zwGqb9DY}\\
        (Dostęp: 29.12.2023)
    \item YouTube - Signing \& Versioning Android Apps\\
        Przydatny poradnik do podpisywania aplikacji wewnątrz pipeline'a.\\
        \href{youtube.com/watch?v=s1grtSSIRVA}{youtube.com/watch?v=s1grtSSIRVA}
        (Dostęp: 29.12.2023)
    \item ZipAlign - narzędzie Android SDK\\
        Opis i sposób użycia narzędzia \verb|zipalign| \\
        \href{developer.android.com/tools/zipalign}{developer.android.com/tools/zipalign}\\
        (Dostęp: 29.12.2023)
    \item Importowanie bibliotek za pomocą mechanizmu refleksji\\
        Krótka prezentacja sposobu importowania dynamicznych bibliotek wewnątrz uruchomionej aplikacji \\
        \href{learn.microsoft.com/en-us/dotnet/framework/reflection-and-codedom/how-to-load-assemblies-into-the-reflection-only-context}%
        {https://learn.microsoft.com/en-us/dotnet/framework/reflection-and-codedom/how-to-load-assemblies-into-the-reflection-only-context}\\
        (Dostęp: 29.12.2023)
    \item Material Symbols - biblioteka darmowych ikon od Google\\
        W mojej aplikacji skorzystałem z kilku ikon udostępnianych przez firmę posiadającą i stale 
        rozwijającą Androida, czyli właśnie Google. Mnogość możliwości ich personalizacji 
        oraz wygoda korzystania (możliwość pobrania całej paczki lub tylko jej części, 
        dostępność formatów wektorowych lub ich rastrowych odpowiedników) sprawiła, 
        że z przyjemnością użyję jej ponownie.
        \href{fonts.google.com/icons}{fonts.google.com/icons}
        (Dostęp: 29.12.2023)
\end{itemize}
