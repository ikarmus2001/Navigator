\section{Motywacja}


Zastosowanie narzędzi wspomagających tworzenie, testowanie, wydawanie i opiekę nad produktem znacząco wpływa na całe rozwiązanie.
Do ich zalet należy przyspieszenie prac, poprawa jakości finalnego produktu oraz odciążenie pracowników z wykonywania powtarzalnych czynności 
(które są powodem \\najczęstszych błędów i niedopatrzeń).
Skorzystanie z ich dobrodziejstw skutkuje wyraźnym poprawieniem komfortu pracy, wydajności zespołu i przede wszystkim jakości kodu.
