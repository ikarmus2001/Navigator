\subsection{Czym jest CI/CD?}
Angielskie \textbf{\textit{Continous Integration / Continous Delivery}} (czyli ciągła integracja i dystrybucja oprogramowania) 
to określenie na zestaw narzędzi automatyzujących dostarczanie programów wraz z ich rozwojem. 
Pełne rozwiązanie, które samodzielnie wypełnia wszystkie nasze kroki wydawcze, nazywamy \textbf{\textit{pipeline}}m (z ang. rurociąg).
Jego istotą jest zapewnienie wydajnego i stabilnego przepływu danych w ramach wypracowanej listy kroków, 
które każdorazowo zostaną wykonane w ten sam sposób~\footnote[1]{
    \label{determinismFootnote}
    Z dokładnością do środowiska, w którym skrypt zostanie uruchomiony, 
    oraz sposobu działania narzędzi (Patrz determinizm kompilatora~\cite{compilerDeterminism})
}, aby otrzymać zamierzony efekt.

\subsubsection{Zalety Pipeline'ów}
Do jego zalet zaliczamy:
\begin{itemize}
    \item Wyeliminowanie błędów i niedopatrzeń człowieka na etapie wydawczym
    \item Przyspieszenie działania procesu
    \item Zwiększenie dostępności 
    \item Determinizm wydania~\footref{determinismFootnote}
    \item Uproszczona analiza szczegółów procesu
    \item Odciążenie zespołu programistycznego z wiedzy o szczegółach procesu wydania
    \item Bezpieczeństwo
\end{itemize}

O ile sam pipeline możemy przedstawić i wykonać jako skrypt terminalowy, 
to jednak dysponujemy już nowoczesnymi narzędziami, które wprowadzają kolejny poziom abstrakcji.
U podstaw są to dokładnie takie same wywołania z linii poleceń, ale opakowanie ich w interfejs graficzny 
lub prostszy język (najczęściej jest to YAML~\cite{yamlDefinition}) upraszcza ich konfigurowanie, 
obniża trudność uczenia się ich obsługi oraz zazwyczaj pozwala na szybszą analizę wyników ich wywołań.

Jako przykład takiego zadania możemy podać skorzystanie .NET Core SDK - o ile musimy poznać możliwości SDK,
to wywołanie ręczne przekazywanie danych wejściowych do terminala zostaje zastąpione modułem, 
dzięki któremu jesteśmy odciążeni z wpisywania słów kluczowych, a jedynie wybieramy je z rozwijanej listy 
lub opisujemy nasze wymagania w skrótach.

Ważnym aspektem jest również bezpieczeństwo - wrażliwe dane, takie jak certyfikaty używane do podpisywania aplikacji czy
hasła dostępowe do serwerów powinny być udostępnione jak najmniejszej ilości osób.

W momencie gdy każdy z deweloperów uaktualnia aplikację ręcznie, musi dysponować tymi informacjami,
zaś korzystając z pipeline'a taki dostęp mają wyłącznie wyselekcjonowane osoby zarządzające procesem.
Zmniejszenie dostępności tych informacji sprawia, że ryzyko wykradnięcia tych danych jest niemal zerowe.

Przykładowo duża firma, która tworzy popularną przeglądarkę internetową, zostaje okradnięta z certyfikatów i haseł - 
będąc w posiadaniu tak istotnych danych, ktoś może stworzyć własną aplikację, która przykładowo wykrada personalia użytkowników, 
a następnie podszyć się pod prawdziwą aplikację. 
Tego rodzaju straty prowadzą do bardzo poważnych konsekwencji, na co nie możemy sobie pozwolić.

\subsubsection{Kompromisy i wady}
Istnieje kilka zagadnień, które należy rozważyć podczas planowania integracji pipeline'a z naszym projektem:
\begin{enumerate}
    \item Ilość czasu potrzebnego do przygotowania konfiguracji
    \item Czy pipeline ma działać w chmurze, czy na własnym urządzeniu
\end{enumerate}

Zwykle przygotowanie konfiguracji pipeline'a jest skomplikowane oraz \\%
czasochłonne, więc do tego zadania zatrudnia się specjalistów\todo{}. \\
Chociaż początkowo może wydawać się to zbędnym nakładem pracy, z upływem czasu zwraca się on z nawiązką.
Jeżeli nasz projekt ma być rozwijany przez więcej osób, w większej perspektywie czasowej, 
to oprócz jego standardowych zalet, zyskujemy również spokój współautorów, 
którzy nie muszą znać szczegółów procesu wydawczego i mogą skupić się na tworzeniu jego zawartości.

Spotkałem się z opiniami, że przywiązanie do jednego dostawcy narzędzia negatywnie wpływa na elastyczność 
takiego pipeline'a - w przypadku potrzeby zmiany narzędzia CI/CD mogą pojawić się komplikacje, 
przykładowo różnice w składni, limity przechowywania poufnych danych, sposób działania~\footnote[2]{
    Niekompatybilność przekazywania zmiennych, konwencje nazewnicze, dostępność szablonów zadań
} itp.
Aby zabezpieczyć się przed takimi trudnościami, należy przeznaczyć dodatkowe nakłady pracy na stworzenie 
modularnych skryptów, które będą łatwe w użyciu.
Sprawa sekretów jest znacznie bardziej skomplikowana - większość firm stosuje mechanizmy uniemożliwiające pobranie już 
zabezpieczonych danych, więc o ile nie przechowujemy ich kopii samodzielnie, może okazać się to istotnym utrudnieniem w migracji.

W zależności od naszych możliwości finansowych oraz zaopatrzeniowych, możemy wynająć maszynę wirtualną
lub zainstalować odpowiednie oprogramowanie CI/CD na własnej maszynie. 
Jeżeli wolimy nie martwić się o sprawy sprzętowe, wygodniejszym rozwiązaniem jest chmura,
ale jeśli dysponujemy sprzętem, to prawdopodobnie skorzystanie z niego będzie tańszą opcją. 
\todo[size=\scriptsize]{czy ja to muszę jakoś potwierdzać? czy to wystarczy?}

