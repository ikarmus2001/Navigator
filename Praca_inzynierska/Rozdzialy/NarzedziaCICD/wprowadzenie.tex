\subsection{Wprowadzenie}
Zaraz po stworzeniu software'u, najważniejszym etapem jest możliwość szybkiego i bezpiecznego dostarczenia go do odbiorców.
W najmniejszych projektach po skompilowaniu aplikacji zazwyczaj wystarczyło skopiowanie plików
i przeniesienie ich na końcowe urządzenie. O ile jest to najprostsze i najbardziej oczywiste rozwiązanie,
jest ono zdecydowanie najmniej wydajne na dłuższą metę.

Wydanie nowej wersji programu możemy w uproszczony sposób przedstawić jako cykl na poniższym rysunku~\ref{fig:cyklZmian}.
Za każdym razem gdy zostaną wprowadzone jakiekolwiek zmiany, należy przygotować nową wersję (kompilacja),
umieścić ją w odpowiedniej lokalizacji (czy to będzie sklep z aplikacjami, strona internetowa czy nośnik cyfrowy),
a następnie poinformować o tym użytkownika (przykładowo wyświetlić powiadomienie o dostępności aktualizacji).


\begin{figure}[!htp]
    \centering
    \begin{tikzpicture}[
        minimum height=1cm,
        block/.style={rectangle, draw=black!60, fill=green!30, thick, rounded corners=2mm}
        ]
            \node[block] (codeChange) {1. Wprowadzenie zmian w kodzie};
            \node[block] (compilation) [right=of codeChange] {2. Kompilacja rozwiązania};
            \node[block] (publish) [below=of compilation] {3. Publikacja plików};
            \node[block] (newVerInfo) [below=of codeChange] {4. Poinformowanie o nowej wersji};

            \draw[->] (codeChange.east) -- node[anchor=east]{} (compilation.west);
            \draw[->] (compilation.south) -- node[anchor=south]{} (publish.north);
            \draw[->] (publish.west) -- node[anchor=west]{} (newVerInfo.east);
            \draw[dotted] (newVerInfo.north) -- node[anchor=north]{} (codeChange.south);
            
    \end{tikzpicture}
    \caption{Uproszczony proces wydawania nowej wersji oprogramowania}
    \label{fig:cyklZmian}
\end{figure}

Wraz ze wzrostem częstotliwości wprowadzania zmian, ilości urządzeń oraz ograniczeniem dostępu do nich,
potrzeba automatyzacji tego procesu staje się coraz bardziej realna. 
W tym celu wykorzystuje się szeroki wachlarz narzędzi CI/CD. \todo{}