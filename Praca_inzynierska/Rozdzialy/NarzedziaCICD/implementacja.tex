\subsection{Implementacja}
W mojej pracy zdecydowałem się na skorzystanie z platformy Azure~\ref{MS_AzureSection} - 
wpłynęły na nią przyjazny interfejs, łatwość instalacji agenta, jasna struktura tworzenia pipeline'ów 
oraz mnogość instrukcji i poradników dotępnych w Internecie.

Na przykładzie aplikacji ,,Navigator''~\ref{NavigatorAppSection} chciałbym przedstawić proces \\
planowania i konfiguracji pipeline'a.

\subsubsection{Określenie potrzeb i możliwości}
W moim projekcie, oprócz samej kompilacji, postanowiłem zaimplementować etap testowania 
oraz wydania aplikacji na wspomnianym wcześniej GitHub Releases~\cite{githubReleases}.
Na Rys.~\ref{img:JobsOverview} widzimy efekt wykonania takiego pipeline'a z poziomu 
widoku zarządzania pipeline'ami na stronie Microsoft Azure:

\begin{figure}[ht]
    \centering
    \includegraphics[scale=0.5]{JobsOverview.png}
    \caption{Zadania (ang. joby) wykonane przez pipeline}
    \label{img:JobsOverview}
\end{figure}

\subsubsection{Przygotowanie środowiska kompilacyjnego}
Pierwszym krokiem było zainstalowanie specjalnego oprogramowania, 
które wykonuje wszystkie kroki zdefiniowane w pipelinie - Azure Build Agent~\footnote{
    \href{https://learn.microsoft.com/en-us/azure/devops/pipelines/agents/agents}{
        Informacje o Agentach
    }
}.
Ze względu na dysponowanie własną, dostępną maszyną zdecydowałem się na tą opcję,
jednak taki sam efekt można osiągnąć opłacając maszyny wirtualne, udostępniane przez Microsoft.

Po krótkiej konfiguracji Agent był gotowy do wykonywania zleconego pipeline'owego skryptu, 
ale należało jeszcze dodać zmienne środowiskowe, które pozwalają na 
skorzystanie z określonych funkcji zadań (przykładowo użycie Javy - paragraf \ref{javaTask}).

\subsubsection{Zdefiniowanie wymaganych akcji}
Utworzenie pakietu mojej aplikacji napisanej w .NET MAUI przebiega za pomocą wywołania kompilatora 
\verb|dotnet| z opcją \cprotect{\href{https://learn.microsoft.com/en-us/dotnet/core/tools/dotnet-publish}}{\verb|publish|}, 
która uruchamia przywrócenie wszystkich zależności (\verb|restore|), 
kompiluje ją (\verb|build|) oraz pakuje do przewidzianego w konfiguracji projektu (\verb|.csproj|) formatu~\cprotect\footnote{%
    Interesującym może okazać się fakt, że aplikację w .NET MAUI kompilujemy w oparciu o projekt, a nie rozwiązanie (\verb|.sln|).
    Jest to spowodowane kolejnością interpretacji plików
}.

Zdecydowałem się na podpisywanie aplikacji osobnym zadaniem ze względu na modularność,
jak również ze względu na chęć ominięcia szczegółów tej operacji wewnątrz samego kodu (tj. w pliku projektu).

\subsubsection{Podejście graficzne}
Moja pierwsza próba konfiguracji została podjęta w ramach graficznego konfiguratora, 
który pozwolił mi na wykonanie (jak się później okazało szkicu) mojego pipeline'a.
Na Rys.~\ref{img:PipelineGuiApproach} i \ref{img:taskGroupMauiPrepare} możemy zobaczyć jego schemat działania:

\begin{figure}[ht]
    \centering
    \begin{minipage}[t]{0.5\textwidth}
        \centering
        \frame{\includegraphics[width=\textwidth]{PipelineGuiApproach_light.png}}
        \caption{Grupa zadań inicjujących środowisko 
            (szczegóły na Rys.~\ref{img:taskGroupMauiPrepare}) 
            oraz publikacja i podpisanie cyfrowe}
        \label{img:PipelineGuiApproach}
    \end{minipage}
    \hfill
    \begin{minipage}[t]{0.47\textwidth}
        \centering
        \frame{\includegraphics[width=\textwidth]{taskGroupMauiPrepare_light.png}}
        \caption{Zawartość grupy zadań inicjujących środowisko}
        \label{img:taskGroupMauiPrepare}
    \end{minipage}
\end{figure}

\newpage

O ile ten pipeline spełniał swoje zadanie na minimalnym poziomie, czyli przygotowywał potrzebne narzędzia 
oraz kompilował i podpisywał aplikację skierowaną na system Android, to jednak jego efektywność i prostota 
utrzymania pozostawiały wiele do życzenia.

Pierwszym z zarzutów jest zbędne wykonanie \verb|dotnet restore| - choć narzut pracy 
jest stosunkowo niewielki, wprowadza kolejny punkt zaczepienia do zadania pytania "czy mogę to zmienić?", 
które mogłaby zadać osoba bliżej niezaznajomiona z działaniem narzędzia.

Drugim problemem okazał się brak wykorzystania zmiennych, które wraz z parametryzacją procesu 
pozwoliłyby na uczynienie go bardziej elastycznym i uniwersalnym. 
Pomimo skorzystania z bezpiecznych zmiennych w procesie podpisywania pliku APK, 
byłem w stanie skorzystać wyłącznie z jednego, wcześniej przygotowanego zestawu danych, 
którego nie mogłem zmienić przed wywołaniem pipeline'a.

% \begin{figure}[hb]
%     \centering
%     \includegraphics[width=0.75\textwidth]{variableGroups.png}
%     \caption{Predefiniowane zmienne przechowujące nazwę oraz hasło podpisujące certyfikat}
%     \label{img:variableGroups}
% \end{figure}
