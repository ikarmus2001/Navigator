\subsection{Wybór systemu CI/CD}

Dobranie odpowiedniego dostawcy oprogramowania CI/CD znacząco wpływa 
na dalsze losy projektu. Najważniejszymi punktami, które \\
powinniśmy podjąć przy omawianiu takiej decyzji to:


\paragraph{Doświadczenie zespołu z danym środowiskiem} 
W zależności od poprze\-dnich doświadczeń i specjalizacji deweloperów, 
możemy obrać wiele ścieżek. Zdarzają się sytuację, gdy większość korzystała 
z konkretnego rozwiązania, mamy w zespole specjalistę DevOps 
- w takich wypadkach możliwość ominięcia etapu wdrażania do nowego narzędzia 
może okazać się bardzo dobrym pomysłem (jeżeli spełnia ono wymagania funkcjonalne).

Jednak w niewielkich projektach i firmach może okazać się, że nie 
mieliśmy okazji pracować z wykorzy\-staniem żadnego z funkcjonalnie zgodnych 
systemów, lub co gorsza z jakimkolwiek oprogramowaniem DevOps. 
W takim wypadku, warto dokładnie rozważyć oferowane na rynku narzędzia, 
aby nie wykonywać dodatkowo frustrującej pracy przy ewentualnej wymuszonej 
migracji między systemami. Osoba oddelegowana do tego zadania powinna wziąć pod 
uwagę niżej wymienione zagadnienia:


\paragraph{Dostępność wymaganych narzędzi}
Każdy projekt ma określone wymogi, które odpowiednio kwalifikują lub 
dyskwalifikują systemy CI/CD z wykorzystania w danym rozwiązaniu.
Najważniejszym z nich jest dostępność i łatwość integracji narzędzi - 
o ile niemalże wszystko jesteśmy w stanie napisać w interfejsie konsolowym, 
wielokrotnie taki narzut pracy został już przez kogoś wykonany, więc nie ma 
potrzeby tracić czas i środki na wypracowanie własnych, niewykluczone że mniej 
wydajnych, algorytmów i skryptów. Mnogość gotowych zadań, przygotowanych 
przez twórców i współpracowników systemu, prawdopodobnie oszczędzi nam 
problemów i przyspieszy implementację pipelineów.


\paragraph{Stabilność, wydajność i "dojrzałość" narzędzia}
Ogólna jakość systemu jest kolejnym ważnym kryterium, którego nie powinniśmy 
pomijać. Te same zadania, ze względu na architekturę czy mnogość możliwości~\footnote{%
    Bardziej skomplikowane środowisko wykonujące skrypty może dodawać niemały narzut 
    pracy - o ile w małych projektach 15 sekund wobec 10 minut czasu działania pipeline'a może 
    nie mieć większego znaczenia, tak w projektach dużych czy budowanych bardzo często 
    może nie być to pomijalna sprawa.
}, 
mogą trwać dłużej, wymagać więcej konfiguracji. 

Istotną sprawą jest również popularność rozwiązania i idąca za tym wielkość społeczności, 
która jest chętna pomóc w razie wszelkich perturbacji - możliwość wymiany 
doświadczeń oraz bezpośrednie wsparcie, w mojej opinii, jest niemalże tak ważne, 
jak dobrze skonstruowana dokumentacja techniczna.
