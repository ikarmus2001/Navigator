\section{Objaśnienie narzędzi}

\subsection{Visual Studio}

Visual Studio to zintegrowane środowisko programistyczne (ang. IDE), czyli połączenie edytora tekstu,
kompilatora oraz debuggera, jak również wielu narzędzi przydatnych podczas rozwijania oprogramowania
(uproszczony refactoring, analiza zużycia zasobów, rozszerzone debugowanie). Pakiety narzędzi dostępne są dla wielu
języków programowania, co czyni z z niego jed

W moim projekcie wykorzystałem wersję 2022 - na dzień wydania pracy jest to najnowsza edycja,
pozwala na szybką konfigurację środowiska do celów kompilacyjnych.

\subsection{Język C\#}

C\# ~\cite{csharpDocs} (z ang. "See Sharp", pol. "Si Szarp") to wydajny język obiektowy wysokiego poziomu, stworzony przez Microsoft.
Należy do rodziny języków C i jest uniwersalny w zastosowaniu.

\subsection{MVVM}

\subsection{.NET MAUI}
https://learn.microsoft.com/en-us/dotnet/maui/

\subsection{Microsoft Azure}
https://learn.microsoft.com/en-us/azure/?product=devops

\subsection{Fluent Assertions}
https://github.com/fluentassertions
https://fluentassertions.com/


\subsection{*Biblioteka pathfinding*}
<Tutaj skrótowo opiszę skąd pobrałem bibliotekę, skrótowo opiszę działanie wykorzystywanego algorytmu>

\subsection{GIT}
Rozproszony system kontroli wersji pozwala na pracę dowolnej ilości osób na dowolnej ilości urządzeń jednocześnie, 
przy ciągłym utrzymaniu aktualności kodu.
GIT z wielu zatwierdzonych części kodu (ang. commitów) podzielonych na gałęzie (ang. branch) tworzy drzewo zmian, 
po którym programiści mogą swobodnie 
poruszać się, zaglądać do dawnych wersji, pomimo braku przechowywania każdej z nich na swoim dysku.
System może być hostowany na własnym serwerze lub chmurze (przykładowo wykorzystany w tym projekcie 
GitHub~\cite{github,Halaczkiewicz_SMCEBI_Navigator_GitHub}), 
ale przechowywanie repozytorium lokalnie jest również możliwe.
