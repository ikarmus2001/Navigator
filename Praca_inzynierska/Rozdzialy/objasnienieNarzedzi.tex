\section{Objaśnienie narzędzi}

\subsection{Visual Studio}

Visual Studio \cite{visualStudio} to zintegrowane środowisko programistyczne (ang. IDE), 
czyli połączenie edytora tekstu, kompilatora oraz debuggera, jak również wielu narzędzi przydatnych podczas rozwijania oprogramowania
(uproszczony refactoring, analiza zużycia zasobów, rozszerzone debugowanie). Pakiety narzędzi dostępne są dla wielu
języków programowania, co czyni z z niego jed

W moim projekcie wykorzystałem wersję 2022 - na dzień wydania pracy jest to najnowsza edycja,
pozwala na szybką konfigurację środowiska do celów kompilacyjnych.
% Prawdopodobnie do przeredagowania


\subsection{Język C\#}

C\# ~\cite{csharpDocs} (z ang. wym. "See Sharp", pol. "Si Szarp") to wydajny język obiektowy wysokiego poziomu, 
stworzony przez Microsoft.
Należy do rodziny języków C i jest uniwersalny w zastosowaniu.
% Prawdopodobnie ten opis należy uzupełnić

\subsection{MVVM}
MVVM to akronim Model-View-Viewmodel (w tłumaczeniu Model - Widok - Model widoku), czyli programistycznego wzorca projektowego.
Polega na wprowadzeniu separacji pomiędzy modelem danych i sposobem ich obróbki, widokami je reprezentującymi, oraz operacjami pomiędzy nimi. 
Jego główną zaletą jest możliwość oddzielenia ogólnych założeń programu od szczegółów integracji z platformą 
- w projekcie~\cite{Halaczkiewicz_SMCEBI_Navigator_GitHub} jest to wyraźnie zarysowane dzięki przyjętej 
konwencji nazewniczej plików (postfix -Model, -ViewModel lub -View).
Na rysunku~\ref{fig:MVVM_diagram} możemy zaobserwować zależność komunikacji klas stworzonych z wykorzystaniem właśnie tego wzorca.

\begin{figure}[!htp]
    \centering
    \begin{tikzpicture}[
        minimum height=1cm,
        block/.style={rectangle, draw=black!60, fill=green!30, thick, rounded corners=2mm}
        ]
            \node[block] (Model) {Model};
            \node[block] (ViewModel) [right=of Model] {ViewModel};
            \node[block] (View) [right=of ViewModel] {View};
            
            \draw[->] (Model.east) -- node[anchor=east]{text} (ViewModel.west);
    \end{tikzpicture}
    \caption{Zależności między klasami w MVVM~\cite{mvvm}}
    \label{fig:MVVM_diagram}
\end{figure}



\subsection{.NET MAUI}
.NET MAUI \cite{mauiDefinition} to platforma programistyczna (ang. framework) umożliwiająca
kompilację natywnych aplikacji wieloplatformowych. Jest to następca Xamarina ~\cite{xamarin},
czyli poprzedniego rozwiązania Microsoftu do tych samych zastosowań. Najważniejszymi różnicami
jest obsługa projektów - Xamarin wymagał osobnych projektów na każdy system operacyjny, zaś MAUI
obsługuje je wewnątrz jednego, uniwersalnego projektu.

Framework umożliwia skorzystanie z kontrolek i interfejsów każdego z systemów za pomocą 
bezpośrednioch odniesień do jego API, a następnie na etapie kompilacji wszelkie odniesienia są
przebudowywane na odpowiednie interfejsy każdej platformy.




\subsection{Microsoft Azure}
Microsoft Azure \cite{azure} to internetowa platforma oferująca szeroki wachlarz narzędzi i usług
z zakresu tworzenia, rozwoju i dystrybucji oprogramowania. Za jej pomocą możemy wynająć moc obliczeniową
w chmurze Microsoftu - począwszy od baz danych, aż po maszyny wirtualne. Platforma oferuje również
pełne środowisko programistyczne - DevOps Server, które wspomagaja komunikację, 
formalizujące wprowadzane zmiany, automatycznie testuje, integruje i dystrybuuje rozwiązania.

\subsection{Fluent Assertions}
Fluent Assertions \cite{fluentassertions} to narzędzie dodające do IDE specjalne rozszerzenia
metod, które wykorzystywane są podczas testowania oprogramowania przed jego wydaniem.
Jego przewagą nad standardowymi rozwiązaniami jest większy wybór i czytelniejsze
nazewnictwo funkcji, jak również wsparcie aktywnej społeczności.


\subsection{*Biblioteka pathfinding*}
<Tutaj skrótowo opiszę skąd pobrałem bibliotekę, skrótowo opiszę działanie wykorzystywanego algorytmu>

\subsection{GIT}
Rozproszony system kontroli wersji pozwala na pracę dowolnej ilości osób na dowolnej ilości urządzeń jednocześnie, 
przy ciągłym utrzymaniu aktualności kodu.
GIT z wielu zatwierdzonych części kodu (ang. commitów) podzielonych na gałęzie (ang. branch) tworzy drzewo zmian, 
po którym programiści mogą swobodnie 
poruszać się, zaglądać do dawnych wersji, pomimo braku przechowywania każdej z nich na swoim dysku.
System może być hostowany na własnym serwerze lub chmurze (przykładowo wykorzystany w tym projekcie 
GitHub~\cite{github,Halaczkiewicz_SMCEBI_Navigator_GitHub}), 
ale przechowywanie repozytorium lokalnie jest również możliwe.
