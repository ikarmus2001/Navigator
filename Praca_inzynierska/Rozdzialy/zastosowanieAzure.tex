\section{Zastosowanie platformy Microsoft Azure}
% zmiana tytułu rozdziału?

% W tym rozdziale planuję opisać w jaki sposób tworzone są kompletne pipeline'y, 
% dlaczego są one tak istotne (najlepiej na przykładzie anegdotycznym?).
% Po krótkim wstępie zaprezentuję jak skonfigurowany jest mój pipeline,
% na jakie elementy należy zwrócić uwagę, jaki jest efekt końcowy jego działania.

Zaraz po stworzeniu software'u, najważniejszym etapem jest możliwość szybkiego i bezpiecznego dostarczenia go do odbiorców.
% Czy ten fragment nie jest za mało konkretny? Coś takiego może się znaleźć w pracy dyplomowej?
W najmniejszych projektach po skompilowaniu aplikacji zazwyczaj wystarczyło skopiowanie plików
i przeniesienie ich na końcowe urządzenie. O ile jest to najprostsze i najbardziej oczywiste rozwiązanie,
jest ono zdecydowanie najmniej wydajne na dłuższą metę.

Wydanie nowej wersji programu możemy w uproszczony sposób przedstawić jako cykl na poniższym rysunku~\ref{fig:cyklZmian}.
Za każdym razem gdy zostaną wprowadzone jakiekolwiek zmiany, należy przygotować nową wersję (kompilacja),
umieścić ją w odpowiedniej lokalizacji (czy to będzie sklep z aplikacjami, strona internetowa czy nośnik cyfrowy),
a następnie poinformować o tym użytkownika (przykładowo wyświetlić powiadomienie o dostępności aktualizacji).


\begin{figure}[!htp]
    \centering
    \begin{tikzpicture}[
        minimum height=1cm,
        block/.style={rectangle, draw=black!60, fill=green!30, thick, rounded corners=2mm}
        ]
            \node[block] (codeChange) {1. Wprowadzenie zmian w kodzie};
            \node[block] (compilation) [right=of codeChange] {2. Kompilacja rozwiązania};
            \node[block] (publish) [below=of compilation] {3. Publikacja plików};
            \node[block] (newVerInfo) [below=of codeChange] {4. Poinformowanie o nowej wersji};

            \draw[->] (codeChange.east) -- node[anchor=east]{} (compilation.west);
            \draw[->] (compilation.south) -- node[anchor=south]{} (publish.north);
            \draw[->] (publish.west) -- node[anchor=west]{} (newVerInfo.east);
            \draw[dotted] (newVerInfo.north) -- node[anchor=north]{} (codeChange.south);
            
    \end{tikzpicture}
    \caption{Uproszczony proces wydawania nowej wersji oprogramowania}
    \label{fig:cyklZmian}
\end{figure}

Wraz ze wzrostem częstotliwości wprowadzania zmian, ilości urządzeń oraz ograniczeniem dostępu do nich,
potrzeba automatyzacji tego procesu staje się coraz bardziej realna.
Prócz przyspieszenia działania, jej główną zaletą jest ograniczenie błędów i niedopatrzeń człowieka.
Im bardziej złożony jest nasz proces wydawczy, tym więcej czasu i wysiłku wymaga pilnowanie jego poprawności -
zautomatyzowanie go daje gwarancję, że żaden z kroków nie zostanie pominięty lub niepoprawnie wykonany.

Istnieje kilka zagadnień, które należy rozważyć podczas planowania zintegrowania do naszego projektu pipeline'a CI/CD,
czyli podzielonego na kroki procesu, który automatyzuje wspomniane wyżej czynności.
\begin{enumerate}
    \item Ilość czasu potrzebnego do przygotowania konfiguracji
    \item Czy pipeline ma działać w chmurze, czy na własnym urządzeniu
\end{enumerate}

Chociaż początkowo może wydawać się to zbędnym nakładem pracy, z upływem czasu zwraca się on z nawiązką.
Jeżeli nasz projekt ma być rozwijany przez więcej osób, w większej perspektywie czasowej, to oprócz jego standardowych
zalet, zyskujemy również spokój współautorów, którzy nie muszą znać szczegółów procesu wydawczego i mogą skupić się
na tworzeniu jego zawartości.

W zależności od naszych możliwości finansowych oraz zaopatrzeniowych, możemy wynająć maszynę wirtualną
lub zainstalować odpowiednie oprogramowanie CI/CD na własnej maszynie. 
Jeżeli wolimy nie martwić się o sprawy sprzętowe, wygodniejszym rozwiązaniem jest chmura,
ale jeśli dysponujemy sprzętem to skorzystanie z niego będzie tańszą opcją. % czy ja to musze jakoś potwierdzać? czy to wystarczy?


