\section{Podsumowanie i wnioski}
Nowoczesny rozwój oprogramowania jest bardzo dynamiczny, dociera już do nawet 
najmniejszych odbiorców prywatnych czy mikrofirm. W przypadku dużych organów 
skorzystanie z takich usług jest kluczowe.

Dla zapewnienia jak najwyższej jakości usług, wykorzystanie przedstawionych 
narzędzi CI/CD jest bardzo istotne. Problemy dystrybucji, środowisk oraz 
czuwania wprowadzeniem zmian zostają w prosty sposób zautomatyzowane, 
co obniża koszty obsługi i zwiększa dostępność takich serwisów.

Celem tej pracy było zaprezentowanie przygotowanego rozwiązania CI/CD, 
wyjaśnienie powodu jego zastosowania, sposobu działania i najistotniejszych 
elementów składowych pipeline'a. Aplikacja "Navigator" posłużyła jako 
przykład dzięki swoim wieloplatformowym możliwościom, ale również zwróciła 
uwagę na istotność dobrych praktyk programistycznych, takich jak zastosowanie 
wzorców projektowych lub odpowiedniego stworzenia architektury projektu.
W rezultacie cel pracy został osiągnięty.

Aplikacja "Navigator" może być w dalszym stopniu rozwijana, nie tylko pod 
względem naprawy błędów czy zmiany jej wyglądu, ale również dodawania 
nowych funkcjonalności. Skonfigurowany pipeline bez wątpienia ułatwi 
i przyspieszy wprowadzanie zmian, a uporządkowana architektura 
zmniejszy ilość trudności przy czytaniu kodu przez nowych programistów.
W celu zmniejszenia ilości wykonywanych operacji, dobrym pomysłem byłoby 
zaimplementowanie zapisywania wygenerowanych widoków HTML w pamięci 
oraz wprowadzenie systemu wersjonowania map.
W celu zmniejszenia ruchu sieciowego, istnieje również opcja 
dodania obsługi lokalnej kopii biblioteki Leaflet.

Modularna struktura plików \verb|.yaml| pozwala na bezproblemowe 
dodanie kolejnych etapów oraz kroków do procesu wydania, 
a ich czytelność ułatwi diagnozowanie ewentualnych błędów.
